% If you use pdflatex to compile this file, 
\documentclass[10pt,a4paper,oneside,onecolumn]{article}
% Else if you use latex and dvipdfmx to compile this file, 
%\documentclass[10pt,a4paper,oneside,onecolumn,dvipdfmx]{article}
% Else if you use latex and dvipdfm to compile this file, 
%\documentclass[10pt,a4paper,oneside,onecolumn,dvipdfm]{article}
% If you use a DVI viewer, e.g. dviout for Windows, 
%\documentclass[10pt,a4paper,oneside,onecolumn,dviout]{article}

% Use the ICROMA.sty package
\usepackage{ICROMA}
\usepackage{enumerate}

\setcounter{page}{1}

% Insert paper title
\title{Scheduling}
% Insert author name(s)
\author{
	Florian Dahms $^{\text{a}}$, 
	Anna-Lena Frank $^{\text{b}}$, 
	Sebastian K\"uhn $^{\text{b}}$,
	Daniel P\"ohle $^{\text{b,1}}$
}
% Insert affiliation information
\affiliation{
	$^{\text{a}}$ ???, ??? \\
	???, 60??? Frankfurt am Main, Germany \\
	$^{\text{b}}$ neXt lab, I.NMF 32, DB Netz AG \\
	Rotfederring 9, 60327 Franfurt am Main, Germany \\
	$^{\text{1}}$ E-mail: Daniel.Poehle@deutschebahn.com, Phone: +49 (0) 69 265 ???
}

%\usepackage{amsmath,amssymb,amstext}
%\usepackage{graphicx}
%\usepackage{subfigure}
%\usepackage[dvips]{epsfig}
%\usepackage{texdraw}
%\usepackage{color}
%\usepackage{url}
%\usepackage[ruled,vlined]{algorithm2e}
%\usepackage{placeins}
%\usepackage{algorithmic}
%\usepackage{multirow}
%\usepackage{tikz}								% Laedt das TikZ-Package (erlaubt das zeichnen von Graphen)
%\usetikzlibrary{arrows}					% erlaubt Pfeile mit TikZ zu zeichnen
%\usetikzlibrary{shapes}					% erlaubt ??? mit TikZ zu zeichnen
%
%\newtheorem{Theorem}{Theorem}[section]	% Damit sind dann Theorem, Korollar und Lemma definiert.
%\newtheorem{Cor}[Theorem]{Korollar}	    % Die Nummerierung erfolgt Kapitelweise.
%\newtheorem{Lemma}[Theorem]{Lemma}
%\newtheorem{Def}[Theorem]{Definiton}
%\newtheorem{Remark}[Theorem]{Remark}
%\newtheorem{Bsp}[Theorem]{Example}
%
%\DeclareSymbolFont{msbm}{U}{msb}{m}{n}
%\DeclareMathSymbol{\N}{\mathalpha}{msbm}{'116}
%\DeclareMathSymbol{\R}{\mathalpha}{msbm}{'122}
%\DeclareMathOperator*{\argmax}{arg\,max}
%
%%\def\ds{\displaystyle}
%
%%\newcommand{\todo}[1]{\par \vspace{2mm}\centerline{\fbox{\fbox{\bf #1}}}\vspace{2mm}}
%
%\newcommand{\abs}[1]{\ensuremath{ \left\lvert {#1} \right\rvert} }	% Betrag
%\newcommand{\st}{\ensuremath{\text{s.t.}} }		% subject to für Optimierungsproblem
%%\newcommand{\N}{\ensuremath{\mathbb{N}}}			% Mengen N
%\newcommand{\Z}{\ensuremath{\mathbb{Z}}}			% Mengen Z
%%\newcommand{\R}{\ensuremath{\mathbb{R}}}			% Mengen R
%\newcommand{\B}{\ensuremath{\mathbb{B}}}			% Menge B (binaere Zahlen)
%\newcommand{\T}{\ensuremath{\text{T}} }				% T für transponiert
%
%\newcommand\nn{\nonumber}					% Keine Gleichungsnummer
%
%\renewcommand{\d}{\ensuremath{\hspace{-0.7mm}\text{d}} }	% d für Integral dr oder dt
%
%
%\numberwithin{equation}{section}

%\newcommand{\my}[1]{{\bf TODO #1}}
%\newcommand{\eqdach}{\stackrel{\wedge}{=}}

%\renewcommand{\thefootnote}{\fnsymbol{footnote}} % -> footnote is not allowed
%\renewcommand{\qedsymbol}{qed}

%\graphicspath{{Bilder/}}		% path to look for (eps)-pictures



\begin{document}
% Pagenumbering is removed (for proceeding generation)

\maketitle

\begin{abstract} %% Not more than 250 words
In this article, we discuss the optimization of a train scheduling problem as well as its usage via an app.
Numerical experiments are accompanied to show the quality of solutions.
\end{abstract}

% Insert 3 - 5 keywords
\keywords{
	traffic networks, network optimization
}


%%%%%%%%%%%%%%%%%%%%%%%%%%%%%%%%%%%%%%%%%%%%%%%%%%%%%%%%%%%%%%%%%%%%%%%%%%%%%%%%%%%%%%%
%%%%%%%%%%%%%%%%%%%%%%%%%%%%%%%%%%%%%%%%%%%%%%%%%%%%%%%%%%%%%%%%%%%%%%%%%%%%%%%%%%%%%%%
\section{Extended Abstract}
\label{chap:extAbs}

\subsection{Problem and Objectives}
We consider two related train scheduling problems. On the one hand every year a timetable of preordered freight trains needs to be scheduled, such that all orders are satisfied, while minimizing travel times of each train. This problem will be refered to as the ``Netzfahrplan''-problem. On the other hand we need to satisfy ad-hoc orders which are coming in throughout the year. This constitutes an online optimization problem where each requests again needs to be satisfied such that its travel time has to be minimized. In order to provide a good customer service we need to be able to create a scheduled train for each incoming request in less than 3 minutes. This will allow the algorithm to be part of a new digital product called the ``Click \& Ride''-App, therefore we will refer to this problem as the C\&R-problem.

Due to the incoming requests throughout the year it is not sufficient to only minimize the travel times of the trains. We also need to schedule each train in a capacity preserving way such that the likelyhood of beeing able to satisy future requests remains high. In order to optimize the available capacity we use so-called ``Systemtrassen'' as a measure of capacity in the railroad network (TODO: literature reference for Systemtrassen). Systemtrassen are preconstructed train schedules for certain parts of the infrastructure. Incoming requests are then assembled partially from Systemtrassen and partially from parts which are calculated ``on-the-fly''.
In this way we transform the process of train scheduling from a microscopic ``make-to-order'' to a mesoscopic``assemble-to-order'' system.

In our numerical examples we show that we are able to increase the available capacity, i.e., the number of possible trains being scheduled, while reducing the average travel time. Due to the automation of the entire process we are additionally able to present an app which is able to provide individual train schedules within three minutes. We also present a use case for this app as a customer friendly tool for sales of train schedules in the ``last minute'', i.e.,\ 72 hours before departure of the train.

\begin{itemize}
  \item assemble to order for capacity planning
  \item 3\%, 5\%, 3 min C\&R-App (TODO: Wollen wir die 3 und 5\% Ziele nennen, oder gehen wir auf Nummer sicher und lassen konkrete Zahlen raus?)
\end{itemize}

\subsection{literature}
\begin{itemize}
  \item TAKT
  \item OpSysTra
\end{itemize}

\subsection{Innovation}
\begin{itemize}
  \item first infrastructure manager to use optimization models for freight train timetable planning
  \item customer value: shorter travel time, higher infrastructure capacity and faster response time
  \item first fully automated process in short-term capacity planning from ordering to the actual departure of the train
\end{itemize}

\subsection{Data}
\begin{itemize}
  \item digital infrastructure model
  \item timetable of passenger trains
\end{itemize}
In our numerical experiments we focus on data from the 13th till 15th of November 2013 for the ``Netzfahrplan''-problem and on data from the 13th till 16th of November 2013 for the C\&R-problem. In both cases we use data for the infrastructure of the German railroads as operated by DB Netz from those days. The digital infrastructure model we use as reference is known as ``Spurplan''. We transform this model into a more suitable representation for our purpose using graph structures. As we concentrate on freight trains only, we consider all passenger trains as fixed blockages of the infrastructure. Therefore we reconstructed all \textit{Sperrzeiten} from 13th to 16th of September 2013 from the timetables of those days. Furthermore, we respect all directives (TODO: spezifischer), which were effective at that time in our optimization model.

\subsection{Practical relevance}
\begin{itemize}
  \item release of C\&R-App in 2019 \\
\end{itemize}
As explained earlier the C\&R-App gives the cargo customer a valid timetable after no more than three minutes. As for now the app is working as designed. However, the app is not used by customer yet since it is still in a evaluation phase, where bugs can be find and be fixed. \\
Nevertheless this situation changes next year as the release of the C\&R-App is planned. This has a huge impact on the timetabling procedure as a customer can now expect her timetable within three minutes. Today a manual short-term creation of a timetable may take up to 72 hours. Therefore many customers order their timetable a long time ahead, even though they might not know the exact details of the required train yet, or even if they will actually need the timetable at all at the requested time.
For example, when reloading goods from a ship to a train, changes in the timetable of the ship will affect the actual departure time of the train, which is therefore hard to know far in advance and the expectation might not fit the actual time anymore. In order to provide a good service to the customer, the train will be disposed through the network at its actual departure anyway, but this practice creates instability during the train networks operation.\\
As a conclusion, the new App will allow customers to order their timetables much later when they have better information about their actual demands and gives them a chance to actually stick to their timetable. It also gives us better data for the construction of the timetable (TODO: How?). \\
As you can see our solution is not only academic it also solves real world problems since x\% of the customer requests can be answered automatically. \\
For the future we want to give the customer on top of it the possibiblity to select within three different timetables to give them more options what kind of timetbabe they want. \\
Furthermore, after going live with the C\&R-App we also want to enrole the automatic planning of the "Netzfahrplan" which does the timetable of cargo trains automtically for one year and the planning of the "Baufahrplan" which reschedule the cargo trains due to constructions. \\
The described next steps do not come all at once since we want to learn how the customers react to the new procedure and we want to incorporate their feedback into the ongoing development. With this approach we get a better quality of the timetable step by step together with our customers. \\
This new way of cargo timetable scheduling prepares DB Netze for the future with an increased efficiency since we have less effort for staff and customer. We are not only more efficient with the planning of timetable we also have a higher capacity utilization of the infrastructure and we are able to offer our cargo customer a better timetable since they reach their destination ealier.   \\

 % Between 750 and 2000 words, up to 10 references

% Full paper (not more than 10 pages)
%\section{Introduction}
\label{chap:intro}

Modeling, simulation and optimization of traffic flow networks based on partial differential equations have been investigated intensively during the last years, 
see for instance for an overview~\cite{CoclitePiccoli, FuegenschuhHertyKlarMartin_MIP, HertyKlar2003, HertyKlar2004, HoldenRisebro, KlarKuehneWegener1996, LighthillWhitham}. 

For optimization purposes, different applications such as optimal routing of traffic at intersections~\cite{ FuegenschuhHertyKlarMartin_MIP, HertyKlar2003, HertyKlar2004}, traffic light control~\cite{GoettlichHertyZiegler2012}, evacuation planning~\cite{NHM, Hamacher} and air traffic control~\cite{BayenRaffardTomlin} are of interest. 
Since in all problems the underlying optimization problem is restricted by nonlinear partial differential equations, 
relaxed models with simplified dynamics have been investigated. In this context, two different solution approaches emerge.
On the one hand, nonlinear continuous optimization techniques have been successfully applied. 
To compute the optimum, the first order optimality system is derived and solved by a descent type method, see~\cite{Kelley1999, NashSofer1996, Spellucci1993}. 
On the other hand, suitable discretization of the dynamics leads to linear network flow models that have been widely considered in the field of combinatorial optimization~\cite{Carey2000157, Fleischer_Tardos, ford1958, koehler2005_load}.

In fact, there exist a few research results comparing both optimization tools,
%to balance computability of large networks and correct dynamics described by partial differential equations by using a mixed integer (MIP) approach.
see~\cite{SimonesBuch, DomschkeGeisslerKolbLangMartinMorsi2011, Fuegenschuh_et_al_MIP, FuegenschuhHertyKlarMartin_MIP, SunStrubBayen2007}. 
One might assume that a one-to-one relation between nonlinear continuous and discrete optimization techniques holds true.
This is especially the case when the governing dynamics in the network are linear. 
%in the state (but not in the control) variables. 
Then, appropriate numerical discretizations can be chosen, such that the original optimization problem 
either leads to numerically solving the finite-dimensional optimality system, i.e.\ the so-called {\em discretize-then-optimize approach},
or the interpretation as a mixed-integer programming model (MIP). However, it remains the question of detecting local or global optima. 
We know from the theory of linear programming, in particular the strong duality theorem~\cite{Chvatal1983, NemhauserWolsey1988, Schrijver1986},
that under certain circumstances a global optimum can be reached. This is usually not the case for the adjoint calculus. The solution of the optimality system 
via gradient methods often gets stuck in local optima.

In this work we want to close the gap between the two solution procedures that first appear to be quite varied.
To do so, we start with the traffic flow network model where the evolution of traffic density on roads is governed by
the linearized Lighthill-Whitham-Richards (LWR) equations, see~\cite{DApiceManzoPiccoli2006}.
For the coupling conditions at the intersections, we stick to the ones presented by Coclite-Garavello-Piccoli~\cite{CoclitePiccoli} 
but do not use a predefined distribution matrix since these parameters will be obtained as solutions of the optimization problem.
In a next step, we discretize the full network model in space and time and formally derive the adjoint equations and the mixed-integer
model (MIP) as well. We show the equivalence of both approaches by comparing the dual variables of the MIP and the adjoint variables. 
A formal, detailed proof for a simple network is given. Finally, we compare the optimization results of the MIP and the adjoint equations 
for different networks and objective functions numerically.

The paper is organized as follows: Section \ref{chap:modelling} contains a description of the network and the models used for describing flow on roads. 
Section \ref{chap:optimization_problem} introduces the optimization problem to be considered. Herein, we also present the main theorem of this article, which we prove in the following sections. 
In section \ref{chap:adjoints}, the adjoint equations for the former optimization problem are derived. % from the Lagrangian function. 
Thereafter, the mixed-integer model is introduced in section \ref{chap:mip}. In section~\ref{chap:MIP_dual}, we conclude
the proof of our main theorem. Numerical examples are given in section~\ref{chap:numerics}. 




%\section{Traffic Flow Network Model}
\label{chap:modelling}

%\section{Optimization Problem}
\label{chap:optimization_problem}

%\section{Mixed Integer Program (MIP)}
\label{chap:mip}


%\section{Numerics}
\label{chap:numerics}

%\section{Outlook}
\label{chap:outlook}
The presented approach of freight train timetable creation prepares DB Netz AG for the future by making better use of the infrastructure and reducing manual workload. Furthermore, this approach is a way to offer timetables faster and in better quality than nowadays.

\textcolor{blue}{Additionally, we fully achieved our goals concerning the automation of the planing process for short term timetables as the app described previously is going live in May 2019.}

For the future, we plan to provide a choice set of up to three different timetables within the app, for the customer to decide which fits best. Up to now, we only consider one day for the planning, so a possible limitation might result from the extension to a whole year, e.g.\ the amount of data generated or the requirement to create homogenous timetables. Nevertheless, we are currently working on this extension to cover multiple days in a simultaneous slot assignment.
Additionally we want to use the presented techniques for additional use cases, e.g.\ the creation of short term timetables and the handling of construction sites. This requires that we extend the algorithms from freight to passenger train scheduling.

%%%%%%%%%%%%%%%%%%%%%%%%%%%%%%%%%%%%%%%%%%%%%%%%%%%%%%%%%%%%%%%%%%%%%%%%%%%%%%%
\section*{Acknowledgements}
This work was financially supported by BMI, Project DigiKap, FKZ ???.
%%%%%%%%%%%%%%%%%%%%%%%%%%%%%%%%%%%%%%%%%%%%%%%%%%%%%%%%%%%%%%%%%%%%%%%%%%%%%%%

%%%%%%%%%%%%%%%%%%%%%%%%%%%%%%%%%%%%%%%%%%%%%%%%%%%%%%%%%%%%%%%%%%%%%%%%%%%%%%%
\begin{thebibliography}{99}
	\setlength{\itemsep}{0\parskip}
	
	\bibitem[Bailey(1995)]{Bailey:1995} 
	Bailey,~C.~(ed.), 1995. 
	{\em European Railway Signalling}, 
	Institution of Railway Signal Engineers (IRSE), 
	A\&C Black, London.
	
	\bibitem[Goverde and Hansen(2000)]{Goverde:2000a} 
	Goverde,~R.M.P., Hansen,~I.A., 2000. 
	``TNV-prepare: analysis of Dutch railway operations 
	based on train detection data", 
	In: Allan,~J., Hill,~R.J., Brebbia,~C.A., Sciutto,~G., Sone,~S.~(eds.), 
	{\em Computers in Railways VII}, 
	pp.~779--788, 
	WIT Press, Southampton.
	
	\bibitem[Goverde and Soto~y~Koelemeijer(2000)]{Goverde:2000b} 
	Goverde,~R.M.P., Soto~y~Koelemeijer,~G., 2000. 
	{\em Performance Evaluation of Periodic Railway Timetables: 
		Theory and Algorithms},
	TRAIL Studies in Transportation Science, 
	Delft University Press, Delft.
	
	\bibitem[Hansen(1999)]{Hansen:1999} 
	Hansen,~I.A., 1999. 
	``Seamless container transport by rail", 
	{\em Rail International}, 
	vol.~30, pp.~7--16.
	
	\bibitem[Huckin and Olsen(1991)]{Huckin:1991} 
	Huckin,~T.N., Olsen,~L.A., 1991. 
	{\em Technical Writing and Professional Communication 
		for Nonnative Speakers of English}, 
	2nd Edition, McGraw-Hill, New York.
	
	\bibitem[Serafini and Ukovich(1989)]{Serafini:1989} 
	Serafini,~P., Ukovich,~W., 1989. 
	``A mathematical model for periodic scheduling problems", 
	{\em SIAM Journal on Discrete Mathematics}, 
	vol. 2, pp. 550-581. 
	http://dx.doi.org/10.1137/0402049
	
	\bibitem[Tamura et~al.(2013)]{Tamura:2013} 
	Tamura,~K., Tomii,~N., Sato,~K., 2013. 
	``An optimal rescheduling algorithm from passengers' viewpoint 
	based on Mixed Integer Programming formulation", 
	In: {\em Proceedings of The 5th International Seminar 
		on Railway Operations Modelling and Analysis (RailCopenhagen2013)}, 
	Lyngby, Denmark.
\end{thebibliography}
%%%%%%%%%%%%%%%%%%%%%%%%%%%%%%%%%%%%%%%%%%%%%%%%%%%%%%%%%%%%%%%%%%%%%%%%%%%%%%%

%%%%%%%%%%%%%%%%%%%%%%%%%%%%%%%%%%%%%%%%%%%%%%%%%%%%%%%%%%%%%%%%%%%%%%%%%%%%%%%%%%%%%%%
%%%%%%%%%%%%%%%%%%%%%%%%%%%%%%%%%%%%%%%%%%%%%%%%%%%%%%%%%%%%%%%%%%%%%%%%%%%%%%%%%%%%%%%
\end{document}