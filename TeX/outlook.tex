\section{Outlook}
\label{chap:outlook}
The presented approach of freight train timetable creation prepares DB Netz AG for the future both by making better use of the infrastructure and by increasing efficiency as there is less effort for staff and customer. Furthermore, this approach is a way to offer timetables faster and in better quality than nowadays.

For the future, we plan to provide a choice set of up to three different timetables within the app, for the customer to decide which fits best. Up to now, we only consider one day for the planning, so a possible limitation might be the amount of data generated by the extension to a whole year. Nevertheless, we are currently working on this extension to cover multiple days in a simultaneous slot assignment.
Additionally, we want to make use of the implemented techniques to provide timetables for other processes than the described use cases of creation of short term timetables and the annual timetable like handling of construction sites. Therefore, we need to extend the applicability of the algorithms from freight train scheduling to passenger train scheduling.