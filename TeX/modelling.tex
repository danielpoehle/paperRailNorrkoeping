\section{Methods of Konstruktion and Belegung}
\label{chap:methods}
In this section we describe our modelling approach to automatically compute timetables. It is divided into two parts: First we present the \emph{Konstruktion} (Section~\ref{chap:Konstruktion}), which "constructs" Systemtrassen based on the work of \cite{O:2009}. Then, in Section~\ref{chap:Belegung}, we describe the process of \emph{Belegung}, where the train paths are assigned to the Systemtrassen yiedling a timetable, which is based on the idea of Nachtigall (e.g. \cite{NO:2014, N:1998, N:2015}).

\subsection{Konstruktion}
\label{chap:Konstruktion}

\subsection{Belegung}
\label{chap:Belegung}