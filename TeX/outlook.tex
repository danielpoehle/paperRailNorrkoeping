\section{Outlook}
\label{chap:outlook}
The presented approach of freight train timetable creation prepares DB Netze for the future both by increasing efficiency as there is less effort for staff and customer and by making better use of the infrastructure. Furthermore, this approach is a way to offer timetables faster and in better quality than nowadays.

For the future, we plan to provide a choice set of up to three different timetables within the app, for the customer to decide which fits best. \textcolor{red}{\textbf{STREICHEN:?}We also plan to incorporate the feedback of our customers into the ongoing development.}
Additionally, we want to make use of the implemented techniques to provide timetables for other processes than the described use cases of creation of short term timetables and the Netzfahrplan like handling of construction sites. Therefore, we need to extend the applicability of the algorithms from freight train scheduling to passenger train scheduling.