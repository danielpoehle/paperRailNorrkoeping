% If you use pdflatex to compile this file,
\documentclass[10pt,a4paper,oneside,onecolumn]{article}
% Else if you use latex and dvipdfmx to compile this file,
%\documentclass[10pt,a4paper,oneside,onecolumn,dvipdfmx]{article}
% Else if you use latex and dvipdfm to compile this file,
%\documentclass[10pt,a4paper,oneside,onecolumn,dvipdfm]{article}
% If you use a DVI viewer, e.g. dviout for Windows,
%\documentclass[10pt,a4paper,oneside,onecolumn,dviout]{article}

% Use the ICROMA.sty package
\usepackage{ICROMA}
\usepackage{enumerate}

\setcounter{page}{1}

% Insert paper title
\title{Transforming automatic scheduling in a working application for a railway infrastructure manager}
% Insert author name(s)
\author{
	Florian H.W. Dahms $^{\text{a}}$,
	Anna-Lena Frank $^{\text{b}}$,
	Sebastian K\"uhn $^{\text{b}}$,
	Daniel P\"ohle $^{\text{b,1}}$
}
% Insert affiliation information
\affiliation{
	$^{\text{a}}$ Vulpes AI GmbH \\
	Textorstrasse 97, 60596 Frankfurt am Main, Germany \\
	$^{\text{b}}$ neXt Lab, I.NMF 32, DB Netz AG \\
	Rotfederring 9, 60327 Franfurt am Main, Germany \\
	$^{\text{1}}$ E-mail: Daniel.Poehle@deutschebahn.com, Phone: +49 (0) 69 265 48267
}

%\usepackage{amsmath,amssymb,amstext}
%\usepackage{graphicx}
%\usepackage{subfigure}
%\usepackage[dvips]{epsfig}
%\usepackage{texdraw}
%\usepackage{color}
%\usepackage{url}
%\usepackage[ruled,vlined]{algorithm2e}
%\usepackage{placeins}
%\usepackage{algorithmic}
%\usepackage{multirow}
%\usepackage{tikz}								% Laedt das TikZ-Package (erlaubt das zeichnen von Graphen)
%\usetikzlibrary{arrows}					% erlaubt Pfeile mit TikZ zu zeichnen
%\usetikzlibrary{shapes}					% erlaubt ??? mit TikZ zu zeichnen
%
%\newtheorem{Theorem}{Theorem}[section]	% Damit sind dann Theorem, Korollar und Lemma definiert.
%\newtheorem{Cor}[Theorem]{Korollar}	    % Die Nummerierung erfolgt Kapitelweise.
%\newtheorem{Lemma}[Theorem]{Lemma}
%\newtheorem{Def}[Theorem]{Definiton}
%\newtheorem{Remark}[Theorem]{Remark}
%\newtheorem{Bsp}[Theorem]{Example}
%
%\DeclareSymbolFont{msbm}{U}{msb}{m}{n}
%\DeclareMathSymbol{\N}{\mathalpha}{msbm}{'116}
%\DeclareMathSymbol{\R}{\mathalpha}{msbm}{'122}
%\DeclareMathOperator*{\argmax}{arg\,max}
%
%%\def\ds{\displaystyle}
%
%%\newcommand{\todo}[1]{\par \vspace{2mm}\centerline{\fbox{\fbox{\bf #1}}}\vspace{2mm}}
%
%\newcommand{\abs}[1]{\ensuremath{ \left\lvert {#1} \right\rvert} }	% Betrag
%\newcommand{\st}{\ensuremath{\text{s.t.}} }		% subject to für Optimierungsproblem
%%\newcommand{\N}{\ensuremath{\mathbb{N}}}			% Mengen N
%\newcommand{\Z}{\ensuremath{\mathbb{Z}}}			% Mengen Z
%%\newcommand{\R}{\ensuremath{\mathbb{R}}}			% Mengen R
%\newcommand{\B}{\ensuremath{\mathbb{B}}}			% Menge B (binaere Zahlen)
%\newcommand{\T}{\ensuremath{\text{T}} }				% T für transponiert
%
%\newcommand\nn{\nonumber}					% Keine Gleichungsnummer
%
%\renewcommand{\d}{\ensuremath{\hspace{-0.7mm}\text{d}} }	% d für Integral dr oder dt
%
%
%\numberwithin{equation}{section}

%\newcommand{\my}[1]{{\bf TODO #1}}
%\newcommand{\eqdach}{\stackrel{\wedge}{=}}

%\renewcommand{\thefootnote}{\fnsymbol{footnote}} % -> footnote is not allowed
%\renewcommand{\qedsymbol}{qed}

%\graphicspath{{Bilder/}}		% path to look for (eps)-pictures



\begin{document}
% Pagenumbering is removed (for proceeding generation)

\maketitle

\begin{abstract} %% Not more than 250 words
In this article, we present a practical approach for the optimized creation of railway timetables. The algorithms are intended to be used by Deutsche Bahn, Germanys largest railway infrastructure provider. We show how our methods can be used, both for creating a timetable in advance and for answering ad-hoc requests coming in via a digital app.
Numerical experiments are provided to show that our solution exceeds manual timetabling in terms of capacity usage, travel times and the time taken for creating the timetable.
\end{abstract}

% Insert 3 - 5 keywords
\keywords{
	traffic networks, network optimization
}


%%%%%%%%%%%%%%%%%%%%%%%%%%%%%%%%%%%%%%%%%%%%%%%%%%%%%%%%%%%%%%%%%%%%%%%%%%%%%%%%%%%%%%%
%%%%%%%%%%%%%%%%%%%%%%%%%%%%%%%%%%%%%%%%%%%%%%%%%%%%%%%%%%%%%%%%%%%%%%%%%%%%%%%%%%%%%%%
\section{Extended Abstract}
\label{chap:extAbs}

\textbf{Denke, dass die Unterkapitel weg sollten. Der extended abstract sollte ein Fließtext sein. Literatur muss dann über die einzelnen Teile verteilt eingebaut werden.}

\subsection{Problem and Objectives}
We consider two related train scheduling problems. On the one hand every year a timetable of preordered freight trains needs to be scheduled, such that all orders are satisfied, while minimizing travel times of each train. This problem will be referred to as the ``Netzfahrplan''-problem. On the other hand we need to satisfy ad-hoc orders which are coming in throughout the year. This constitutes an online optimization problem, where each request again needs to be satisfied, such that its travel time is minimized. In order to provide a good customer service we need to be able to create a scheduled train for each incoming request in less than 3 minutes. This will allow the algorithm to be part of a new digital product called the ``Click \& Ride''-App, therefore we will refer to this problem as the C\&R-problem.

Due to the incoming requests throughout the year it is not sufficient to only minimize the travel times of the trains. We also need to schedule each train in a capacity preserving way such that the likelihood of being able to satisfy future requests remains high. In order to optimize the available capacity we use so-called ``Systemtrassen'' as a measure of capacity in the railroad network (\textbf{TODO: literature reference for Systemtrassen}). Systemtrassen are preconstructed train schedules for certain parts of the infrastructure. Incoming requests are then assembled partially from Systemtrassen and partially from parts which are calculated ``on-the-fly''.
In this way we transform the process of train scheduling from a microscopic ``make-to-order'' to a mesoscopic ``assemble-to-order'' system.

In our numerical examples we show that we are able to increase the available capacity, i.e., the number of possible trains being scheduled, while reducing the average travel time. Due to the automation of the entire process we are additionally able to present an app which is able to provide individual train schedules within three minutes. We also present a use case for this app as a customer friendly tool for sales of train schedules in the ``last minute'', i.e.,\ 72 hours before departure of the train.

\begin{itemize}
  \item assemble to order for capacity planning
  \item 3\%, 5\%, 3 min C\&R-App (TODO: Wollen wir die 3 und 5\% Ziele nennen, oder gehen wir auf Nummer sicher und lassen konkrete Zahlen raus?)
\end{itemize}

\subsection{literature}
\begin{itemize}
  \item TAKT
  \item OpSysTra
\end{itemize}

\subsection{Innovation}
\begin{itemize}
  \item first infrastructure manager to use optimization models for freight train timetable planning
  \item customer value: shorter travel time, higher infrastructure capacity and faster response time
  \item first fully automated process in short-term capacity planning from ordering to the actual departure of the train
\end{itemize}

\subsection{Data}
\begin{itemize}
  \item digital infrastructure model
  \item timetable of passenger trains
\end{itemize}
In our numerical experiments we focus on data from the 13th until 15th of November 2013 for the ``Netzfahrplan''-problem and on data from the 13th until 16th of November 2013 for the C\&R-problem. In both cases we use data for the infrastructure of the German railroads as operated by DB Netz from those days. The digital infrastructure model we use as reference is known as ``Spurplan'' (\textbf{Add reference?}). We transform this model into a more suitable representation for our purpose using graph structures. As we concentrate on freight trains only, we consider all passenger trains as fixed blockages of the infrastructure. Therefore we reconstructed all \textit{Sperrzeiten} \textbf{Check: bekannter Fachbegriff?} from 13th to 16th of September 2013 from the timetables of those days. Furthermore, we respect all directives (TODO: spezifischer), which were effective at that time in our optimization model.

\subsection{Practical relevance}
\begin{itemize}
  \item release of C\&R-App in 2019 \\
\end{itemize}
As explained earlier the C\&R-App gives the cargo customer a valid timetable after no more than three minutes. As for now the app is working as designed. However, the app is not used by customer yet since it is still in a evaluation phase, where bugs can be found and fixed. \\
Nevertheless this situation changes next year as the release of the C\&R-App is planned. This is going to have an huge impact on the timetabling procedure as customers can then expect their timetables within three minutes. Today a manual short-term creation of a timetable may take up to 72 hours. Therefore many customers order their timetables a long time ahead, even though they might not know the exact details of the required train yet, or even if they will actually need the timetable at all at the requested time.
For example, when reloading goods from a ship to a train, changes in the timetable of the ship will affect the actual departure time of the train, which is therefore hard to know far in advance and the expectation might not fit the actual time anymore. In order to provide a good service to the customer, the train will be disposed through the network at its actual departure anyway, but this practice creates instability during the operation of the train network.\\
As a conclusion, the new app will allow customers to order their timetables much later when they have better information about their actual demands and gives them a chance to actually stick to their timetable. Furthermore, due to the better information of the actual requested train, better data for the construction of the timetable is provided, yielding a more appropriate use of infrastructure and thus better timetables for both the infrastructure operating company and the customer (\textbf{TODO: How? $\rightarrow$ solved?}). \\
Consequently, our solution is not only academic it also solves real world problems since x\% (\textbf{add correct number here}) of the customer requests can be answered automatically. \\
For the future we plan to provide a choice set of three different timetables within the app, for the customer to decide which option fits best. \\
Furthermore, after going live with the C\&R-App, we also want to make the automatic planning of the "Netzfahrplan" available, which automatically creates timetables of cargo trains for one year and the planning of the "Baufahrplan" which reschedule the cargo trains due to constructions. \textbf{Baufahrplan bisher überhaupt nicht erwähnt? Warum jetzt hier im Lookout?} \\
The described next steps do not come all at once since we want to learn how the customers react to the new procedure and we want to incorporate their feedback into the ongoing development. With this approach we get a better quality of the timetable step by step together with our customers. \\
This new way of cargo timetable scheduling prepares DB Netze for the future with an increased efficiency since we have less effort for staff and customer. We are not only more efficient in the planning of timetables, but also have a higher capacity utilization of the infrastructure as well as are able to offer our cargo customer a better timetable since they reach their destination earlier.   \\

 % Between 750 and 2000 words, up to 10 references

% Full paper (not more than 10 pages)
%\section{Introduction}
\label{chap:intro}
%
The usual process of creating a timetable, especially for freight trains, is a manual \emph{make-to-order} process. But as \cite{FP:2014} already pointed out: It is necessary to overcome the manual process to be able to find global optimal solutions in an industrialized creation of timetables as the amount of data to be processed is steadily increasing. Additionally, in order to improve the efficiency of the process of creating a timetable as well as making better use of the available infrastructure it is convenient to transform the process to an \emph{assemble-to-order} process.

This transformation as well as different approaches for an automatic creation of rail freight timetables have been widely discussed in the literature. Planning of slots was first introduced for cyclic timetables using a PESP by \cite{N:1998} and extended by \cite{O:2009}. \cite{G:2012} showed that the PESP can also be encoded as a SAT problem which leads to significantly lower computation times. For non cyclic timetables, \cite{G:2013} proved that a mixed integer formulation works for practical problem sizes. The train path assignment problem as the second step after planning of slots was introduced by \cite{NO:2014} and extended for optimization with different traffic days \cite{N:2015}. 

In the actual paper, we point out the academic models mentioned above can be transformed into an application in real world optimization with an innovation that is threefold: First, we are able to generate more capacity of infrastructure without actually building it but by making better use of it. Second, we are able to reduce the travel time for most of the scheduled trains by simultaneously considering all requests and using global optimization instead of manual, local optimization. Third, we reduce response times by implementing the first fully automated process in short-term capacity planing from ordering to the actual departure of the train. All resulting in a better customer value.

The organization of the paper is as follows. First, we describe the modelling approach in Section~\ref{chap:methods}. After that, we describe the set of data used for our experiments (Section~\ref{chap:data}) and discuss the use case as well as the experiments in Section~\ref{chap:useCases}. We close with an outlook (Section~\ref{chap:outlook}) to possible extensions of our approach in the future.

%\section{Modelling approach \emph{\textcolor{blue}{[Methods of] train path construction and assignment}}}
\label{chap:methods}
In this section we briefly describe our modelling approach to automatically create timetables. It is divided into two main parts: First, we calculate train paths, which are located on the most frequently used part of the infrastructure (Section~\ref{chap:Konstruktion}). Therefore, we extend the idea of \cite{O:2009}. The second part is the train path assignment (Section~\ref{chap:Belegung}), based on the idea of \cite{N:1998, N:2015, NO:2014}, where some single train paths are assigned to a complete timetable.


\subsection{Train path calculation}
\label{chap:Konstruktion}
\textbf{
\textcolor{red}{TODO by Florian}
\begin{itemize}
	\item Konstruktion kurz beschreiben ($\leq$1Seite)
	\item Routing \textbf{DONE}
	\item snippet creation \textbf{OPEN}
	\item train path calculation \textbf{OPEN}
\end{itemize}
}

This section will give a brief overview of the process used for creating a train path. The entire process consists of three main steps. First, we search one or more different routes through the infrastructure. In the second step we create a network of discrete building blocks (called snippets) along these routes for which we calculate travel and blocking times. Finally, we put these snippets together to form a non conflicting train path.


\subsubsection{Routing}

To reduce the problem size, our first step consists of finding routes that could be relevant for the train. We use an A* algorithm for finding a shortest route, which utilizes geo coordinates for calculating the beeline distance as the lower bound. The infrastructure of Deutsche Bahn is divided into sections called \emph{Betriebstellen}. For each Betriebstelle we keep a list of all possible ways to traverse it. Each possibility is termed a \emph{Fahrweg}. The Fahrwege form a graph where each Fahrweg is a vertex, with directed edges indicating which Fahrweg is a direct successor of another. The edge costs are based on the Fahrwege lengths. But as certain Fahrwege are preferred to others we multiply the cost with factor which is closer to one for more desirable Fahrwege. This way the use of intersections and the use of tracks from the opposite direction can be discouraged. It is this graph we use of finding routes for our train paths.

While exploring the graph the algorithm filters out all paths, which are incompatible with the characteristics of the train.
As it is not always the best option to use the shortest path, we create multiple alternative routes. For searching the subsequent routes, we increase the costs of edges that where used in already found routes. A route is only accepted as a real alternative if it differs from all already calculated one in at least one Betriebstelle. We set an upper limit to the length of alternative routes which is a multiple ($1.3$ in the experiments) of the shortest route found.


\subsubsection{Snippet Creation}

For each route we calculate the travel and blocking times that would be required for a train using the route without intermediate stops. To enable stopping, we search for all tracks that could be used for a stop of the train along the route. A track is only considered for stopping, if it branches off the main route and joins it again within a single Betriebstelle. For each such track we create one snippet leading from the main route to the stop and one from the stop to the main route. These snippets have a length of $7$km, a length that ensures that acceleration and deceleration to and from the main travel speed is always possible within the snippet. For each snippet we again calculate travel and blocking times. In order to connect these snippets with the main route, we cut it into snippets at the points where our stopping snippets branch off it or join it back. In addition to these stop snippets we snippets for alternative non stopping traversals of a Betriebstelle for all tracks that traverse the Betriebstelle similar to the original main route.

This way we get a directed graph of snippets representing the possible ways the train can travel along each route including intermediate stops. Each snippet has travel and blocking times calculated. A path from a source snippet (those without predecessors) to a sink snippet (without successors) will always represent a valid train path (without a specific starting time of the train).


\subsubsection{Train Path Calculation}


\subsection{train path assignment}
\label{chap:Belegung}
\textbf{
	\textcolor{red}{TODO by SK}
\begin{itemize}
	\item Belegung kurz beschreiben ($\leq$1Seite)
	\item EAPkte DONE
	\item Wegesuche Erster Versuch
	\item VNI OPEN
	\item Optimization Erster Versuch
\end{itemize}
}
As a result of the train path calculation described in the previous section (Section~\ref{chap:Konstruktion}), we are provided with a directed graph \G\, in which the Systemtrassen and \emph{\textcolor{red}{Spurplanknoten or BST}} form the arcs and the nodes, respectively. In a process consisting of four steps, this graph is used to assign train paths to the requests made by customers. First, we map the requests of the customer to the graph \G, resulting in so-called \emph{break-in} and \emph{break-out points}. Then, we search for the shortest path using Systemtrassen In the third part, we calculate train paths from the start of a request to the break-in point and from the break-out point to the goal of the request. Finally, we optimize the result for all requests using a column generation approach.

\subsubsection{Calculation of break-in and break-out points \emph{\textcolor{blue}{Mapping onto Systemtrassen}}}
As most customers do not put a request starting or ending at nodes of the graph \G\, (compare Figure~\ref{fig:E-A-Pkte}), we need to map the start of a request to break-in points and break-out points, which are nodes of the graph \G. Therefore, we calculate a number of different route (10 in the experiment), using the A* algorithm described previously. Then, beginning at the start of the route, we consecutively check for each \textcolor{red}{BST?} of each route whether it is associated to a node of the graph \G. If so, we found a break-in point and stop; otherwise the request has to be calculated individually. In order to find the break-out points, we repeat the above process starting from the end of each route.

\subsubsection{Routing on Systemtrassen}
The mapping of a request previously described may not be unique, resulting in a time-dependent multi-source-multi-sink shortest path problem to be solved for each request. Furthermore, due to the technical properties of the requests such as e.g.\ length, acceleration or width, the use of Systemtrassen is restricted depending on the request.
We use standard techniques like time expansion to tackle the time dependency, dummy edges for multi-source and multi-sink problem as well as dynamic filters for the technical properties, reducing the problem to a standard shortest path problem, which we solve using a Dijkstra algorithm.

\subsubsection{Vor-/Nachlauf}

\subsubsection{Optimization}
The process described above is sufficient if only one request hast to be fulfilled at a time like in the use case of the app. But as there is more than one request in the use case of Netzfahrplan the simple assignment of the shortest path for each request leads to conflicts between the train paths. We solve these conflicts using a column generation approach, where in each iteration we generate new train paths which resolve more conflicts than in the iteration before. Our experiments show, that the process terminates within up to 10 hours for sufficiently large problems (compare Section~\ref{chap:Netzfahrplan}).

%\section{Optimization Problem}
\label{chap:optimization_problem}

%\section{Mixed Integer Program (MIP)}
\label{chap:mip}


%\section{Numerics}
\label{chap:numerics}

%\section{Outlook}
\label{chap:outlook}
The presented approach of freight train timetable creation prepares the DB Netze for the future both by increasing efficiency as there is less effort for staff and customer and by making better use of the infrastructure. Furthermore, this approach is a way to offer timetables faster and in better quality than nowadays.

For the future, we plan to provide a choice set of up to three different timetables within the app, for the customer to decide which fits best. \textcolor{red}{\textbf{STREICHEN:?}We also plan to incorporate the feedback of our customers into the ongoing development.}
Additionally, we want to make use of the implemented techniques to provide timetables for other processes than the described use cases of creation of short term timetables and the Netzfahrplan like handling of construction sites. Therefore, we need to extend the applicability of the algorithms from freight train scheduling to passenger train scheduling.

%%%%%%%%%%%%%%%%%%%%%%%%%%%%%%%%%%%%%%%%%%%%%%%%%%%%%%%%%%%%%%%%%%%%%%%%%%%%%%%
\section*{Acknowledgements}
<<<<<<< HEAD
This work was financially supported by BMVI, Project DigiKap.
=======
This work was financially supported by BMI, Project Digitale Kapazit\"atssteigerung, FKZ ???.
>>>>>>> 8f076ee7fe3c69a1e2a90a14330cb01aa49b6753
%%%%%%%%%%%%%%%%%%%%%%%%%%%%%%%%%%%%%%%%%%%%%%%%%%%%%%%%%%%%%%%%%%%%%%%%%%%%%%%

%%%%%%%%%%%%%%%%%%%%%%%%%%%%%%%%%%%%%%%%%%%%%%%%%%%%%%%%%%%%%%%%%%%%%%%%%%%%%%%
\begin{thebibliography}{99}
	\setlength{\itemsep}{0\parskip}

	\bibitem[Bailey(1995)]{Bailey:1995}
	Bailey,~C.~(ed.), 1995.
	{\em European Railway Signalling},
	Institution of Railway Signal Engineers (IRSE),
	A\&C Black, London.
	
	\bibitem[Feil and P\"ohle(2014)]{FP:2014}
	Feil,~M., P\"ohle,~D., 2014.
	{``Why Does a Railway Infrastructure Company Need an Optimized Train Path Assignment for Industrialized Timetabling"},
	International Conference on Operations Research,
	Aachen.
	
	\bibitem[Nachtigall and Opitz(2014)]{NO:2014}
	Nachtigall,~K., Opitz,~J., 2014.
	{``Modelling and Solving a Train Path Assignment Model"},
	International Conference on Operations Research,
	Aachen.
	
	\bibitem[Nachtigall and Opitz(2014)]{N:2015}
	Nachtigall,~K., 2014.
	{``Modelling and Solving a Train Path Assignment Model with Traffic Day Restriction"},
	In: {\em Operations Research Proceedings 2015},
	Springer, Heidelberg.
	
	\bibitem[Nachtigall(1998)]{N:1998}
	Nachtigall,~K., 1998
	{``Periodic Network Optimization and Fixed Interval Timetables"},
	Habilitation thesis,
	Hildesheim.
	
	\bibitem[Opitz(2014)]{O:2009}
	Opitz,~J., 2009
	{``Automatische Erzeugung und Optimierung von Taktfahrplänen in Schienenverkehrsnetzen"},
	Dissertation,
	Dresden.
	
	\bibitem[Großmann et~al.(2012)]{G:2012}
	Gro{\ss}mann,~P., H\"olldobler,~S., Mantey,~N., Nachtigall,~K, Opitz,~J., Steinke,~P., 2012.
	``Solving Periodic Event Scheduling Problems with SAT",
	In: {\em Advanced Research in Applied Artificial Intelligence},
	Springer Berlin Heidelberg.
	
	\bibitem[Großmann et~al.(2013)]{G:2013}
	Gro{\ss}mann,~P., Labinsky,~A., Opitz,~J., Wei{\ss},~R., 2013.
	``Capacity-utilized Integration and Optimization of Rail Freight Train Paths into 24 Hours Timetables",
	In: {\em Proceedings of the 3rd International Conference on Models and Technologies for Intelligent Transportation Systems 2013},
	TUDpress, Dresden.

	\bibitem[Goverde and Hansen(2000)]{Goverde:2000a}
	Goverde,~R.M.P., Hansen,~I.A., 2000.
	``TNV-prepare: analysis of Dutch railway operations
	based on train detection data",
	In: Allan,~J., Hill,~R.J., Brebbia,~C.A., Sciutto,~G., Sone,~S.~(eds.),
	{\em Computers in Railways VII},
	pp.~779--788,
	WIT Press, Southampton.

	\bibitem[Goverde and Soto~y~Koelemeijer(2000)]{Goverde:2000b}
	Goverde,~R.M.P., Soto~y~Koelemeijer,~G., 2000.
	{\em Performance Evaluation of Periodic Railway Timetables:
		Theory and Algorithms},
	TRAIL Studies in Transportation Science,
	Delft University Press, Delft.

	\bibitem[Hansen(1999)]{Hansen:1999}
	Hansen,~I.A., 1999.
	``Seamless container transport by rail",
	{\em Rail International},
	vol.~30, pp.~7--16.

	\bibitem[Huckin and Olsen(1991)]{Huckin:1991}
	Huckin,~T.N., Olsen,~L.A., 1991.
	{\em Technical Writing and Professional Communication
		for Nonnative Speakers of English},
	2nd Edition, McGraw-Hill, New York.

	\bibitem[Serafini and Ukovich(1989)]{Serafini:1989}
	Serafini,~P., Ukovich,~W., 1989.
	``A mathematical model for periodic scheduling problems",
	{\em SIAM Journal on Discrete Mathematics},
	vol. 2, pp. 550-581.
	http://dx.doi.org/10.1137/0402049

	\bibitem[Tamura et~al.(2013)]{Tamura:2013}
	Tamura,~K., Tomii,~N., Sato,~K., 2013.
	``An optimal rescheduling algorithm from passengers' viewpoint
	based on Mixed Integer Programming formulation",
	In: {\em Proceedings of The 5th International Seminar
		on Railway Operations Modelling and Analysis (RailCopenhagen2013)},
	Lyngby, Denmark.
\end{thebibliography}
%%%%%%%%%%%%%%%%%%%%%%%%%%%%%%%%%%%%%%%%%%%%%%%%%%%%%%%%%%%%%%%%%%%%%%%%%%%%%%%

%%%%%%%%%%%%%%%%%%%%%%%%%%%%%%%%%%%%%%%%%%%%%%%%%%%%%%%%%%%%%%%%%%%%%%%%%%%%%%%%%%%%%%%
%%%%%%%%%%%%%%%%%%%%%%%%%%%%%%%%%%%%%%%%%%%%%%%%%%%%%%%%%%%%%%%%%%%%%%%%%%%%%%%%%%%%%%%
\end{document}