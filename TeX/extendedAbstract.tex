\section{Extended Abstract}
\label{chap:extAbs}

\subsection{Problem and Objectives}
We consider two related train scheduling problems. On the one hand every year a timetable of preordered freight trains needs to be scheduled, such that all orders are satisfied, while minimizing travel times of each train. This problem will be refered to as the ``Netzfahrplan''-problem. On the other hand we need to satisfy ad-hoc orders which are coming in throughout the year. This constitutes an online optimization problem where each requests again needs to be satisfied such that its travel time has to be minimized. In order to provide a good customer service we need to be able to create a scheduled train for each incoming request in less than 3 minutes. This will allow the algorithm to be part of a new digital product called the ``Click \& Ride''-App, therefore we will refer to this problem as the C\&R-problem.

Due to the incoming requests throughout the year it is not sufficient to only minimize the travel times of the trains. We also need to schedule each train in a capacity preserving way such that the likelyhood of beeing able to satisy future requests remains high. In order to optimize the available capacity we use so-called ``Systemtrassen'' as a measure of capacity in the railroad network (TODO: literature reference for Systemtrassen). Systemtrassen are preconstructed train schedules for certain parts of the infrastructure. Incoming requests are then assembled partially from Systemtrassen and partially from parts which are calculated ``on-the-fly''.
In this way we transform the process of train scheduling from ``make-to-order'' to ``assemble-to-order''.

In our numerical examples we show that we are able to increase the available capacity, i.e., the number of possible trains being scheduled, while reducing the average travel time. Due to the automation of the entire process we are additionallz able to present an app which is able to provide individual train schedules within three minutes. We also present a use case for this app as a customer friendly tool for sales of train schedules in the ``last minute'', i.e.,\ 72 hours before departure of the train.

\begin{itemize}
  \item assemble to order for capacity planning
  \item 3\%, 5\%, 3 min C\&R-App (TODO: Wollen wir die 3 und 5\% Ziele nennen, oder gehen wir auf Nummer sicher und lassen konkrete Zahlen raus?)
\end{itemize}

\subsection{literature}
\begin{itemize}
  \item TAKT
  \item OpSysTra
\end{itemize}

\subsection{Innovation}
\begin{itemize}
  \item first infrastructure manager to use optimization models for freight train timetable planning
  \item customer value: shorter travel time, higher infrastructure capacity and faster response time
  \item first fully automated process in short-term capacity planning from ordering to the actual departure of the train
\end{itemize}

\subsection{Data}
\begin{itemize}
  \item digital infrastructure model
  \item timetable of passenger trains
\end{itemize}
In our numerical experiments we focus on data from the 13th till 15th of November 2013 for the ``Netzfahrplan''-problem and on data from the 13th till 16th of November 2013 for the C\&R-problem. In both cases we use data for the infrastructure of the German railroads as operated by DB Netz from those days. The digital infrastructure model we use as reference is known as ``Spurplan''. We transform this model into a more suitable representation for our purpose using graph structures. As we concentrate on freight trains only, we consider all passenger trains as fixed blockages of the infrastructure. Therefore we reconstructed all \textit{Sperrzeiten} from 13th to 16th of September 2013 from the timetables of those days. Furthermore, we respect all directives, which were effective at that time in our optimization model.

\subsection{Practical relevance}
\begin{itemize}
  \item release of C\&R-App in 2019 \\
\end{itemize}
As explained earlier the C\&R-App gives the cargo customer a valid timetable after no more than three minutes. As for now the app is working as designed. However, the app is not used by customer yet since it is still in a evaluation phase, where bugs can be find and be fixed. \\
Nevertheless this situation changes next year as the release of the C\&R-App is planned. This has a huge impact on the timetabling procedure as a customer can now expect her timetable within three minutes. Today a manual short-term creation of a timetable may take up to 72 hours. Therefore many customers order their timetable a long time ahead, even though they might not know the exact details of the required train yet, or even if they will actually need the timetable at all at the requested time.
For example, when reloading goods from a ship to a train, changes in the timetable of the ship will affect the actual departure time of the train, which is therefore hard to know far in advance and the expectation might not fit the actual time anymore. In order to provide a good service to the customer, the train will be disposed through the network at its actual departure anyway, but this practice creates instability during the train networks operation.\\
As a conclusion, the new App will allow customers to order their timetables much later when they have better information about their actual demands and gives them a chance to actually stick to their timetable. It also gives us better data for the construction of the timetable (TODO: How?). \\
As you can see our solution is not only academic it also solves real world problems since x\% of the customer requests can be answered automatically. \\
For the future we want to give the customer on top of it the possibiblity to select within three different timetables to give them more options what kind of timetbabe they want. \\
Furthermore, after going live with the C\&R-App we also want to enrole the automatic planning of the "Netzfahrplan" which does the timetable of cargo trains automtically for one year and the planning of the "Baufahrplan" which reschedule the cargo trains due to constructions. \\
The descriped next steps do not come all at once since we want to learn how the customer react to the new procedure and we want to have the chance to be flexible due to their reactions. With this approach we get a better quality of the timetable step by step together with our customers. \\
This new way of cargo timetable construktion prepares DB Netze for the future with an increased efficiency since we have less effort for staff and customer. We are not only more efficient with the planning of timetable we also have a higher capacity utilization of the infrastructure and we are able to offer our cargo customer a better timetable since they reach their destination ealier.   \\

