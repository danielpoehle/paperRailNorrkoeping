\documentclass[final]{siamltex}
%	[12pt,	% Schriftgroesse
%	a4paper,	% Papiergroesse
%	twoside,	% zweiseitig
%	DIV11,		% Verhaeltnis von Rand zu Text (groesser=mehr Text)
%	BCOR10mm,	% Korrekturen fuer Druck
%	abstracton,	% Ueberschrift bei abstract&acknowledgments
%	openright,	% auf rechter Seite anfangen
%	footexclude=false,	% ???
%	draft]		% Fuer erste Versionen (spaeter final)
%	pointlessnumbers,	% macht die Punkte hinter der Kapitelnummer weg
%	final]		% siehe oben
%	{scrreprt}	% Klasse: deutscher Report
%	{report}		% amerikanischer Report

\usepackage{amsmath,amssymb,amstext}
\usepackage{graphicx}
\usepackage{subfigure}
\usepackage[dvips]{epsfig}
\usepackage{texdraw}
\usepackage{color}
\usepackage{url}
\usepackage[ruled,vlined]{algorithm2e}
\usepackage{placeins}
\usepackage{algorithmic}
\usepackage{multirow} 
\usepackage{tikz}								% Laedt das TikZ-Package (erlaubt das zeichnen von Graphen)
\usetikzlibrary{arrows}					% erlaubt Pfeile mit TikZ zu zeichnen
\usetikzlibrary{shapes}					% erlaubt ??? mit TikZ zu zeichnen

\newtheorem{Theorem}{Theorem}[section]	% Damit sind dann Theorem, Korollar und Lemma definiert.
\newtheorem{Cor}[Theorem]{Korollar}	    % Die Nummerierung erfolgt Kapitelweise.
\newtheorem{Lemma}[Theorem]{Lemma}
\newtheorem{Def}[Theorem]{Definiton}
\newtheorem{Remark}[Theorem]{Remark}
\newtheorem{Bsp}[Theorem]{Example}

\DeclareSymbolFont{msbm}{U}{msb}{m}{n}
\DeclareMathSymbol{\N}{\mathalpha}{msbm}{'116}
\DeclareMathSymbol{\R}{\mathalpha}{msbm}{'122}
\DeclareMathOperator*{\argmax}{arg\,max}

%\def\ds{\displaystyle}

%\newcommand{\todo}[1]{\par \vspace{2mm}\centerline{\fbox{\fbox{\bf #1}}}\vspace{2mm}}

\newcommand{\abs}[1]{\ensuremath{ \left\lvert {#1} \right\rvert} }	% Betrag
\newcommand{\st}{\ensuremath{\text{s.t.}} }		% subject to für Optimierungsproblem
%\newcommand{\N}{\ensuremath{\mathbb{N}}}			% Mengen N
\newcommand{\Z}{\ensuremath{\mathbb{Z}}}			% Mengen Z
%\newcommand{\R}{\ensuremath{\mathbb{R}}}			% Mengen R
\newcommand{\B}{\ensuremath{\mathbb{B}}}			% Menge B (binaere Zahlen)
\newcommand{\T}{\ensuremath{\text{T}} }				% T für transponiert

\newcommand\nn{\nonumber}					% Keine Gleichungsnummer

\renewcommand{\d}{\ensuremath{\hspace{-0.7mm}\text{d}} }	% d für Integral dr oder dt


\numberwithin{equation}{section}

%\newcommand{\my}[1]{{\bf TODO #1}}
%\newcommand{\eqdach}{\stackrel{\wedge}{=}}

\renewcommand{\thefootnote}{\fnsymbol{footnote}}
%\renewcommand{\qedsymbol}{qed}

\graphicspath{{Bilder/}}		% path to look for (eps)-pictures

\title{Scheduling}
%\thanks{This work was financially supported by BMI, Project DigiKap, FKZ ???.}}
\author{
Florian Dahms\thanks{???, ???, Frankfurt, Germany} ({\tt dahms@???.de})
\and Anna-Lena Frank\thanks{I.NMF 32, DB Netz AG, Frankfurt am Main, Germany} ({\tt Anna-Lena.Frank@deutschebahn.com, Sebastian.Se.Kuehn@deutschebahn.com, Daniel.Poehle@deutschebahn.com})
\and Sebastian K\"uhn\footnotemark[2]
\and Daniel P\"ohle\footnotemark[2]}


\begin{document}

\maketitle

\begin{abstract} %% Not more than 250 words
In this article, we discuss the optimization of
a linearized traffic flow network model based on conservation laws.
We present two solution approaches. One relies on the classical Lagrangian formalism
(or adjoint calculus), whereas another one uses a discrete mixed-integer framework.
We show how both approaches are related to each other theoretically.
Numerical experiments are accompanied to show the quality of solutions.
\end{abstract}

\begin{keywords}
traffic networks, network optimization
\end{keywords}

\begin{AMS}
90B20, 90C11
% 90B20 Traffic problems, 90C11 Mixed integer programming
\end{AMS}

\pagestyle{myheadings}
\thispagestyle{plain}
\markboth{F. DAHMS, A-L. FRANK, S. K\"UHN AND D. P\"OHLE}{SCHEDULING}

%%%%%%%%%%%%%%%%%%%%%%%%%%%%%%%%%%%%%%%%%%%%%%%%%%%%%%%%%%%%%%%%%%%%%%%%%%%%%%%%%%%%%%%
%%%%%%%%%%%%%%%%%%%%%%%%%%%%%%%%%%%%%%%%%%%%%%%%%%%%%%%%%%%%%%%%%%%%%%%%%%%%%%%%%%%%%%%

\section{Introduction}
\label{chap:intro}

Modeling, simulation and optimization of traffic flow networks based on partial differential equations have been investigated intensively during the last years, 
see for instance for an overview~\cite{CoclitePiccoli, FuegenschuhHertyKlarMartin_MIP, HertyKlar2003, HertyKlar2004, HoldenRisebro, KlarKuehneWegener1996, LighthillWhitham}. 

For optimization purposes, different applications such as optimal routing of traffic at intersections~\cite{ FuegenschuhHertyKlarMartin_MIP, HertyKlar2003, HertyKlar2004}, traffic light control~\cite{GoettlichHertyZiegler2012}, evacuation planning~\cite{NHM, Hamacher} and air traffic control~\cite{BayenRaffardTomlin} are of interest. 
Since in all problems the underlying optimization problem is restricted by nonlinear partial differential equations, 
relaxed models with simplified dynamics have been investigated. In this context, two different solution approaches emerge.
On the one hand, nonlinear continuous optimization techniques have been successfully applied. 
To compute the optimum, the first order optimality system is derived and solved by a descent type method, see~\cite{Kelley1999, NashSofer1996, Spellucci1993}. 
On the other hand, suitable discretization of the dynamics leads to linear network flow models that have been widely considered in the field of combinatorial optimization~\cite{Carey2000157, Fleischer_Tardos, ford1958, koehler2005_load}.

In fact, there exist a few research results comparing both optimization tools,
%to balance computability of large networks and correct dynamics described by partial differential equations by using a mixed integer (MIP) approach.
see~\cite{SimonesBuch, DomschkeGeisslerKolbLangMartinMorsi2011, Fuegenschuh_et_al_MIP, FuegenschuhHertyKlarMartin_MIP, SunStrubBayen2007}. 
One might assume that a one-to-one relation between nonlinear continuous and discrete optimization techniques holds true.
This is especially the case when the governing dynamics in the network are linear. 
%in the state (but not in the control) variables. 
Then, appropriate numerical discretizations can be chosen, such that the original optimization problem 
either leads to numerically solving the finite-dimensional optimality system, i.e.\ the so-called {\em discretize-then-optimize approach},
or the interpretation as a mixed-integer programming model (MIP). However, it remains the question of detecting local or global optima. 
We know from the theory of linear programming, in particular the strong duality theorem~\cite{Chvatal1983, NemhauserWolsey1988, Schrijver1986},
that under certain circumstances a global optimum can be reached. This is usually not the case for the adjoint calculus. The solution of the optimality system 
via gradient methods often gets stuck in local optima.

In this work we want to close the gap between the two solution procedures that first appear to be quite varied.
To do so, we start with the traffic flow network model where the evolution of traffic density on roads is governed by
the linearized Lighthill-Whitham-Richards (LWR) equations, see~\cite{DApiceManzoPiccoli2006}.
For the coupling conditions at the intersections, we stick to the ones presented by Coclite-Garavello-Piccoli~\cite{CoclitePiccoli} 
but do not use a predefined distribution matrix since these parameters will be obtained as solutions of the optimization problem.
In a next step, we discretize the full network model in space and time and formally derive the adjoint equations and the mixed-integer
model (MIP) as well. We show the equivalence of both approaches by comparing the dual variables of the MIP and the adjoint variables. 
A formal, detailed proof for a simple network is given. Finally, we compare the optimization results of the MIP and the adjoint equations 
for different networks and objective functions numerically.

The paper is organized as follows: Section \ref{chap:modelling} contains a description of the network and the models used for describing flow on roads. 
Section \ref{chap:optimization_problem} introduces the optimization problem to be considered. Herein, we also present the main theorem of this article, which we prove in the following sections. 
In section \ref{chap:adjoints}, the adjoint equations for the former optimization problem are derived. % from the Lagrangian function. 
Thereafter, the mixed-integer model is introduced in section \ref{chap:mip}. In section~\ref{chap:MIP_dual}, we conclude
the proof of our main theorem. Numerical examples are given in section~\ref{chap:numerics}. 




\section{Traffic Flow Network Model}
\label{chap:modelling}

\section{Optimization Problem}
\label{chap:optimization_problem}

\section{Mixed Integer Program (MIP)}
\label{chap:mip}


\section{Numerics}
\label{chap:numerics}

%\section{Outlook}
\label{chap:outlook}
The presented approach of freight train timetable creation prepares DB Netz AG for the future by making better use of the infrastructure and reducing manual workload. Furthermore, this approach is a way to offer timetables faster and in better quality than nowadays.

\textcolor{blue}{Additionally, we fully achieved our goals concerning the automation of the planing process for short term timetables as the app described previously is going live in May 2019.}

For the future, we plan to provide a choice set of up to three different timetables within the app, for the customer to decide which fits best. Up to now, we only consider one day for the planning, so a possible limitation might result from the extension to a whole year, e.g.\ the amount of data generated or the requirement to create homogenous timetables. Nevertheless, we are currently working on this extension to cover multiple days in a simultaneous slot assignment.
Additionally we want to use the presented techniques for additional use cases, e.g.\ the creation of short term timetables and the handling of construction sites. This requires that we extend the algorithms from freight to passenger train scheduling.

%%%%%%%%%%%%%%%%%%%%%%%%%%%%%%%%%%%%%%%%%%%%%%%%%%%%%%%%%%%%%%%%%%%%%%%%%%%%%%%
\section*{Acknowledgements}
This work was financially supported by BMI, Project DigiKap, FKZ ???.
%%%%%%%%%%%%%%%%%%%%%%%%%%%%%%%%%%%%%%%%%%%%%%%%%%%%%%%%%%%%%%%%%%%%%%%%%%%%%%%

%%%%%%%%%%%%%%%%%%%%%%%%%%%%%%%%%%%%%%%%%%%%%%%%%%%%%%%%%%%%%%%%%%%%%%%%%%%%%%%
%\bibliographystyle{plain}
\bibliographystyle{siam}
\bibliography{Bibliographie}
%%%%%%%%%%%%%%%%%%%%%%%%%%%%%%%%%%%%%%%%%%%%%%%%%%%%%%%%%%%%%%%%%%%%%%%%%%%%%%%

%%%%%%%%%%%%%%%%%%%%%%%%%%%%%%%%%%%%%%%%%%%%%%%%%%%%%%%%%%%%%%%%%%%%%%%%%%%%%%%%%%%%%%%
%%%%%%%%%%%%%%%%%%%%%%%%%%%%%%%%%%%%%%%%%%%%%%%%%%%%%%%%%%%%%%%%%%%%%%%%%%%%%%%%%%%%%%%
\end{document}