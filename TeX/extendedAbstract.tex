\section{Extended Abstract}
\label{chap:extAbs}

\subsection{problem and objectives}
\begin{itemize}
  \item assemble to order for capacity planning
  \item 3\%, 5\%, 3 min C\&R-App
\end{itemize}

\subsection{literature}
\begin{itemize}
  \item TAKT
  \item OpSysTra
\end{itemize}

\subsection{innovation}
\begin{itemize}
  \item first infrastructure manager to use optimization models for freight train timetable planning
  \item customer value: shorter travel time, higher infrastructure capacity and faster response time
  \item first fully automated process in short-term capacity planning from ordering to the actual departure of the train
\end{itemize}

\subsection{data}
\begin{itemize}
  \item digital infrastructure model
  \item timetable of passenger trains
\end{itemize}

\subsection{practical relevance}
\begin{itemize}
  \item release of C\&R-App in 2019 \\
  As in the subsection problem and objectives explained the C\&R-App will give the customer a response with a selection of three possible 
  timetable after three minutes. As for now the app is working as design. However, the app is not used by the customer yet since it is still in a 
  evaluation phase where bugs can be find and will be fixed. \\
  Nevertheless this phase will change next year as the release of the C\&R-App is planned next years. This will have an extremely impact of the 
  timetable procedure since then the customer will get his timetable within three minutes. Furthermore he has the opportunity to select within three 
  different options. 

  \item first step in automatization of timetable planning and traffic management \\
  1 C\&R-App for the Gelegenheitsverkehr
  2 automisation of the Netzfahrlplan 
  3 Baufahrplan automatization 

  \item prepares DB Netz for the future with an increased efficiency (higher capacity and less effort for staff and customers) \\
  1 higher capacity due to Systemtrassen
  2 timetable can be produced within six hours
  3 less effort for for staff and customers
\end{itemize}
