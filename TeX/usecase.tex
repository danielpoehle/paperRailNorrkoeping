\section{Use Case: Click \& Ride App and Netzfahrplan}
\label{chap:useCases}
In this section, we describe the use cases which were used to proof that our approach for the automatic opitimized creation of railway timetables performes. For this we set ourselves three goals: 

\begin{itemize}
	\item[1)] An customer's request suppose to be proceed after three minutes with the Click\&Ride App
	\item[2)] The Systemtrassen are suppose to increase the capazity of the german railway netzwork
	\item[3)] With the Belegung on the Systemtrassen we minimize the time customer spend within the network
\end{itemize}

\subsection{Click \& Ride App}
\label{chap:CnR}

\subsection{Netzfahrplan}
\label{chap:Netzfahrplan}
The new process of creating the Netzfahrplan will be devided into two steps. The first step is the manuall creation of the 
public transport which takes place during the day and the scound step is the automatic creation of the freight transport during the night. This interaction takes place every day during the period of the creation of the Netzfahrplan. As a consequence the manuall and the automatic generation can interact with each other every day. \\
\\
\textbf{\textcolor{red}{Falls noch Platz, könnte hier das Schaubild hinzufügen}}


