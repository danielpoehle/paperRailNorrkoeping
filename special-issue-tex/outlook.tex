\section{Outlook}
\label{chap:outlook}
The presented approach of freight train timetable creation prepares DB Netz AG for the future by making better use of the infrastructure and reducing manual workload. Furthermore, this approach is a way to offer timetables faster and in better quality than nowadays.
For the future, we plan to provide a choice set of up to three different timetables within the app, for the customer to decide which fits best. Up to now, we only consider one day for the planning, so a possible limitation might result from the extension to a whole year, e.g.\ the amount of data generated or the requirement to create homogenous timetables. Nevertheless, we are currently working on this extension to cover multiple days in a simultaneous slot assignment.
Additionally we want to use the presented techniques for additional use cases, e.g.\ the creation of short term timetables and the handling of construction sites. This requires that we extend the algorithms from freight to passenger train scheduling.