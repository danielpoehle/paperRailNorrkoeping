\section{Extended Abstract}
\label{chap:extAbs}

\subsection{problem and objectives}
\begin{itemize}
  \item assemble to order for capacity planning
  \item 3\%, 5\%, 3 min C\&R-App
\end{itemize}
In this article, we point out how to transform the process of train scheduling from a make to order system to an assemble to order process. Therefore we apply a mesoscopic modelling approach introducing so-called Systemtrassen as a meassure of capacity in the railroad network. With this model we are able to realize two objectives of optimization. On the one hand we aim to gain more capacity on the tracks, i.e.\ there are more train paths available with our optimization model ompared to manual planing. ON the other hand we optimize the travel time of the generated train paths. In our numerical example we show that we are able to generate 3 \% more capacity and 5 \% faster train paths compared to manual planing. The benefits from the optimization approach are not only twofold but threefold. Due to the transformation to an assemble to order process we are addiontionally able to present an app which is able to provide individual train paths within three minutes. We also present an use case for this app as a customer friendly tool for the last minute, i.e.\ 72 hours before departure of the train, sales of train paths.
\subsection{literature}
\begin{itemize}
  \item TAKT
  \item OpSysTra
\end{itemize}

\subsection{innovation}
\begin{itemize}
  \item first infrastructure manager to use optimization models for freight train timetable planning
  \item customer value: shorter travel time, higher infrastructure capacity and faster response time
  \item first fully automated process in short-term capacity planning from ordering to the actual departure of the train
\end{itemize}

\subsection{data}
\begin{itemize}
  \item digital infrastructure model
  \item timetable of passenger trains
\end{itemize}
For our numerical example we consider the scheduling problem for freight trains in Germany. As the reference date we consider the 14th of November 2013 for the whole timetabling problem and 14th to 16th of November 2013 for the app. In both cases we consider the infrastructure of the German railroads as operated by DB Netz from those day or days, resp.. Therefore we rely on the digital infrastructure model callled Spurplan as reference, which we transform into a more suitable representation for our purpose using graph structures. As we concentrate on freight trains only, we consider all passenger trains as fixed blockages of the infrastructure. Therefore we reconstructed all \textit{Sperrzeiten} from 14th to 16th of September 2013 from the timetables of those days. Furthermore, we respect all directives, which were effective at that time in our optimization model.
\subsection{practical relevance}
\begin{itemize}
  \item release of C\&R-App in 2019 \\
\end{itemize}
As explained earlier the C\&R-App gives the cargo customer a valid timetable after an average of three minutes. As for now the app is working as design. However, the app is not used by customer yet since it is still in a evaluation phase where bugs can be find and be fixed. \\
Nevertheless this situation changes next year as the release of the C\&R-App is planned. This has an extremely impact of the timetable procedure since then the customer gets his timetable within three minutes. Today a short-term creation of a timetable takes up to 72 hours, since we have to take all the details of the construction of one train into account and the construction is done manually. Therefore customer order their timetable a long time ahead even if they don't exactly know whether they need the timetable at the requested time or not. For example, when goods are relaoded from a ship to a train. Here, it is really hard to find out the actuall departure time of the train and therefore the ordered timetable might not fit to the actuall time anymore. But since the customer ordered a timetable we have to dispose the train through our network which makes the network unstable. \\
As a conclusion, the new App gives not only the customer a chance to actually stick on there timetable it also gives us better data for the construction of a timetable. \\
As you can see our solution is not only academic it also solves real world problems since x\% of the customer requests can be answered automatically. \\
For the future we want to give the customer on top of it the possibiblity to select within three different timetables to give them more options what kind of timetbabe they want. \\
Furthermore, after going live with the C\&R-App we also want to enrole the automatic planning of the "Netzfahrplan" which does the timetable of cargo trains automtically for one year and the planning of the "Baufahrplan" which reschedule the cargo trains due to constructions. \\
The descriped next steps do not come all at once since we want to learn how the customer react to the new procedure and we want to have the chance to be flexible due to their reactions. With this approach we get a better quality of the timetable step by step together with our customers. \\
This new way of cargo timetable construktion prepares DB Netze for the future with an increased efficiency since we have less effort for staff and customer. We are not only more efficient with the planning of timetable we also have a higher capacity utilization of the infrastructure and we are able to offer our cargo customer a better timetable since they reaches there destination ealier.   \\

